\documentclass{article}
\usepackage[utf8]{inputenc}
\usepackage[spanish]{babel}
\title{Entrega 1}
\author{jhoan mateo franco vargas }
\date{\today}

\begin{document}

\maketitle
\renewcommand*\contentsname{Contenido}

\tableofcontents
\newpage


\section{Diseño de una memoria Ram:}

Se desea diseñar el sistema de control de lectura y escritura de una memoria RAM de 32 filas
x 32 columnas, donde cada dato es de 4 bits. Para la primera entrega deberá presentar una
propuesta de diseño de tres circuitos combinables en Logisim:

\begin{enumerate}
    \item Un decodificador que convierte una señal de 5 bits a un único dato (entre 0 y 31) para
la selección de filas y columnas de una memoria RAM, y para el control de lectura/
escritura de la misma.
    \item Un circuito de control con tres entradas y dos salidas:
    \begin{enumerate}
        \item Entrada chip select (CS) funciona como la habilitación de la memoria RAM.
        \item Entrada write enable (WE) para activar la escritura en la memoria.
        \item Entrada output enable (OE) para activar el envío de datos por el bus.
    \end{enumerate}
    Cuando CS y WE están activados, la salida E (escritura) debe activarse. Si WE está activado,
la salida L (lectura) debe permanecer inactiva. Si CS y OE están activados, la salida L (lectura)
debe activarse.
    \item Un circuito conversor para un display de siete segmentos, que se utilizará para visualizar
posteriormente el dato disponible en la memoria RAM.
La entrega consiste en un documento que presente el diseño, cálculos, diagramas, investigación
y demás consideraciones que se hayan tenido para la primera entrega. Además, se debe enviar
un archivo de Logisim con los tres sub-circuitos solicitados.
\end{enumerate}
\newpage
es una nueva pagina


\end{document}

